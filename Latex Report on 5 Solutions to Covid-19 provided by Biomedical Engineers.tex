\documentclass[12pt]{article}
\begin{document}
\title{A Latex report on 5 solutions to Covid-19 provided by Biomedical Engineers}
\maketitle{}

\section{Introduction}
Biomedical engineers are pivotal in the use of technology as part of patient care, from procurement, to maintenance but also, and this is a little less known, working with clinician to produce innovative devices to enable novel treatments.

The shortage of medical equipment and PPE led to governments rushing to buy and stockpile functional devices as well as setting up challenges for their development and production. Hundreds of projects were put together in record times by certified medical device manufacturers, individuals and teams of engineers alike. Information and designs were shared online in a true collaborative effort for anyone to attempt to develop solutions that could be crucial in tackling the pandemic.


\section{5 Solutions to Covid-19}
1.Ventilators:-Patients who cannot breathe spontaneously need to be put on a ventilator. Ventilators are capable of replacing the breath function and patients in an advanced state of respiratory distress are usually intubated and sedated at the beginning of the treatment.

Ventilators are capable of replacing the breath function and patients in an advanced state of respiratory distress are usually intubated and sedated at the beginning of the treatment. They are complex systems providing the healthcare professionals with a lot of flexibility to adapt the assisted breathing settings and to be able to wean recovering patients off the ventilator gradually.

2.Continuous Positive Airway Pressure (CPAP):-A well-fitted face mask is an essential component of a CPAP system and as such it is quite intrusive. CPAP is only appropriate for patients who are capable of some breathing strength as CPAP effectively opposes some resistance to expiration. Variants exist that either automatically adjust the level of pressure to the patients breathing characteristics (APAP) or have different levels of pressure for inspiration and expiration (BiPAP). CPAP usually supplies (filtered) air to the patient but most masks have a port for supplementing the supply with oxygen.

3.Patient monitoring:-An essential element of the ICU equipment is the monitoring equipment that keeps track of some of the patient vitals especially when they are ventilated and sedated but also during their recovery phase to ensure the regime of ventilation is optimised for their condition. Ventilators already provide their set of patient parameters, but usually patient monitors are separate devices as they continue to be useful after the patient can resume breathing on their own unassisted.Modern patient monitors provide many more patient parameters all the way to breathing waveforms to enable clinicians to fine tune their care of the patients.

4.Oxygen concentrators:-The first form for mild respiratory insufficiency is usually the supply of extra oxygen through a nasal cannula or a more intrusive face mask. Most of the time, the oxygen comes in the form of cylinders, either small for portability or large for stationary patients and longer-term supply.Oxygen concentrators represent an attractive alternative to tanks although this is not typically in use while caring for COVID-19 patients in hospital settings. Oxygen concentrators extract oxygen from the air on demand and supply it directly to the patient. Concentrators come in a variety of sizes from a portable shoulder bag form factor, to higher capacity stationary machines for patients who need oxygen 24/7.

5.Personal protective equipment:-The COVID-19 pandemic has evidenced the fragility of society and the need for effective and practical ways to protect it. For the general public, the use of face masks as personal protection equipment (PPE) remain the most practical line of defence against SARS-CoV-2 as well as other respiratory viral infections.for the wide range of multidisciplinary health care workers more protection is required, as surgical or respirator masks, and these are not intended to be worn for so long as is required in an NHS shift. There is an environmental cost to these disposable items, they do not fit all face shapes, the mask-face seal can be broken while talking, and they apply pressure to the sensitive face skin which can cause discomfort and tissue injury.

\section{Conclusion}
Despite the laudable effort and the many successful projects that stemmed out of it and have been mentioned in this report, many devices that looked technically sound on paper and cost-effective for mass production may not have been up to standard. Certain user requirements and human factors may not have been evident to teams of engineers without clinical experience in highly infectious disease wards and intensive care units. For example, coronavirus patients require specialised, reliable, repeatable and controlled ventilation with compatibility with the many different systems involved in intensive care such as kidney filtration device and drug delivery systems.

It is therefore paramount that end users such as clinicians and nurses are involved from the first stage of design development in order to ensure devices developed are fit for purpose. This is particularly important during a global health crisis, where avoidable delays in production and development can have impact on healthcare workers’ and patient’s wellbeing and society’s ability to respond. 




\end{document}